% このファイルは日本語用です。
% 次の行は変更しないでください。
\documentclass[ja]{2021b}
%%%%%%%%%%%%%%%%%%%%%%%%%%%%%%%%%%%%%%%%%%%%%%%%%%%%%%%%%%%%%%%%
% 講演者についての情報
\PresenterInfo
%%%%%%%%%%%%%%%%%%%%%%%%%%%%%%%%
% 講演数(半角数字)
{1}
%%%%%%%%%%%%%%%%%%%%%%%%%%%%%%%%
% 氏名
{谷口暁星}
%%%%%%%%%%%%%%%%%%%%%%%%%%%%%%%%
% 氏(ひらがな, 氏名が英字の場合はalphabet)
{たにぐち}
%%%%%%%%%%%%%%%%%%%%%%%%%%%%%%%%
% 名(ひらがな, 氏名が英字の場合はalphabet)
{あきお}
%%%%%%%%%%%%%%%%%%%%%%%%%%%%%%%%
% 所属機関(機関名(◯◯大学、◯◯研究所、など)のみ)
{名古屋大学}
%%%%%%%%%%%%%%%%%%%%%%%%%%%%%%%%
% 会員種別(半角英小文字)
%   a=正会員(一般)
%   b=正会員(学生)
%   c=準会員(一般)
%   d=準会員(学生)
%   e=非会員(一般)〔企画セッションのみ〕
%   f=非会員(学生)〔企画セッションのみ〕
{a}
%%%%%%%%%%%%%%%%%%%%%%%%%%%%%%%%
% 会員番号(半角数字4桁)
%   入会申請中の場合、受付番号(半角英数字8文字)
{5892}
%%%%%%%%%%%%%%%%%%%%%%%%%%%%%%%%
% メールアドレス(半角)
{taniguchi@a.phys.nagoya-u.ac.jp}
%%%%%%%%%%%%%%%%%%%%%%%%%%%%%%%%%%%%%%%%%%%%%%%%%%%%%%%%%%%%%%%%
% 講演についての情報
\PaperInfo
%%%%%%%%%%%%%%%%%%%%%%%%%%%%%%%%
% 記者発表(半角英小文字)
%   申請する場合のみ「y」を記入
{}
%%%%%%%%%%%%%%%%%%%%%%%%%%%%%%%%
% 講演分野(半角)
%  [通常セッション]
%   M=太陽
%   N=恒星・恒星進化
%   P1=星・惑星形成(星形成)
%   P2=星・惑星形成(原始惑星系円盤)
%   P3=星・惑星形成(惑星系)
%   Q=星間現象
%   R=銀河
%   S=活動銀河核
%   T=銀河団
%   U=宇宙論
%   V1=観測機器(電波)
%   V2=観測機器(光赤外・重力波・その他)
%   V3=観測機器(X線・γ線)
%   W=コンパクト天体
%   X=銀河形成・進化
%   Y=天文教育・広報普及・その他
%  [企画セッション]
%   Z1=次世代サブミリ波-テラヘルツ地上単一鏡
{Z1}
%%%%%%%%%%%%%%%%%%%%%%%%%%%%%%%%
% 講演形式(半角英小文字)
%   a=口頭講演
%   b=ポスター講演(口頭有)
%   c=ポスター講演(口頭無)
{a}
%%%%%%%%%%%%%%%%%%%%%%%%%%%%%%%%
% キーワード(5つまで)
%   分野Y以外は PASJ keyword list から選択
{methods: data analysis}
{methods: observational}
{methods: statistical}
{}
{}
%%%%%%%%%%%%%%%%%%%%%%%%%%%%%%%%
% 題名
{次世代の地上単一鏡装置開発におけるデータ科学の応用}
%%%%%%%%%%%%%%%%%%%%%%%%%%%%%%%%
% 氏名及び所属(複数の場合は「, 」で区切)
{
    谷口暁星(名古屋大学)
}
%%%%%%%%%%%%%%%%%%%%%%%%%%%%%%%%%%%%%%%%%%%%%%%%%%%%%%%%%%%%%%%%
\begin{document}
%%%%%%%%%%%%%%%%%%%%%%%%%%%%%%%%%%%%%%%%%%%%%%%%%%%%%%%%%%%%%%%%
% 本文開始
%%%%%%%%%%%%%%%%%%%%%%%%%%%%%%%%%%%%%%%%%%%%%%%%%%%%%%%%%%%%%%%%
データ科学は、画像のような``大きな''データやサンプルが不完全な``小さな''データなどから新たな知見を得るためのアプローチとして近年注目され、機械学習やスパースモデリングをはじめとする様々な手法が天文学へ応用されつつある。
次世代の地上単一鏡計画においても、広視野・広波長域の同時分光撮像に対する世界的なニーズからビッグデータ化は不可避であり、従来の``人の手を介した''処理に頼らない観測・データ処理・天体検出の方法論の確立が急務である。
また、多波長観測とのシナジーを前提としたデータ公開の仕組みも重要である。

本講演では、次世代の単一鏡装置開発におけるデータ科学が果たす役割と課題を、実際の応用例を交えて紹介する。
一般的な地上観測では、天体信号は複数の要素(地球大気・アンテナ・受信機・分光計)を通して観測者に届くため、それぞれの要素での信号の劣化が課題である。
この際、天体信号または雑音信号の持つ統計的な性質に即したデータ科学的方法を導入できれば、信号の劣化を抑えることが可能である。
一例として、大気放射の天体信号への重畳は感度を制限する大きな要因である。
本講演では大気放射を自動的に分離するデータ科学的方法を紹介し、次世代の装置開発においてデータ科学が観測感度の向上に必要不可欠であることを示す。

また、先行してビッグデータ化が進む光赤外望遠鏡の例を挙げつつ、観測から公開までのデータの取扱いに関する課題も述べる。
特に、観測データの全てを観測者が手元にダウンロードし解析するという従来の方法は困難になることが予想される。
本講演では、これに代わる方法としてデータ解析や可視化をクラウドベースで行うサイエンスプラットフォームの可能性を紹介し、実現のために必要なデータ形式やデータ処理の開発課題を示す。

%%%%%%%%%%%%%%%%%%%%%%%%%%%%%%%%%%%%%%%%%%%%%%%%%%%%%%%%%%%%%%%%
% 本文終了
%%%%%%%%%%%%%%%%%%%%%%%%%%%%%%%%%%%%%%%%%%%%%%%%%%%%%%%%%%%%%%%%
\end{document}
